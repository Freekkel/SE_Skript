\chapter{Zusammenfassung - Lernziele}
\begin{itemize}
    \item \textbf{Software Engineering ist eine Ingenieursdisziplin, die sich mit allen Aspekten der Softwareproduktion beschäftigt}
    \item High-level activities specifications, Entwicklung, Validierung und Evolution sind Teil aller Softwareprozesse.
    \item Prozesse:
    \begin{itemize}
        \item Wasserfallmodell
        \item V-Modell
        \item Boehm's spiral model
        \item RUP (modernes generisches Prozessmodell)
        \item agile Methoden (Scrum, XP, \dots),\dots
    \end{itemize}
    \item Es gibt \textbf{viele verschiedene Systemformen}, ihre Entwicklung erfordert eigene Software Engineering-Tools und Techniken 
    \item \textbf{Requirements engineering} ist der Prozess der Entwicklung einer Software-Spezifikation 
    \begin{itemize}
        \item Kernpunkte: User- und System Requirements, Funktionale und nicht funktionale Anforderungen, Qualitätseigenschaften der Anforderungen
    \end{itemize}
    \item Design: \textbf{high-level design} (architectural design) und \textbf{low-level design}
    \begin{itemize}
        \item Pattersn als Weg, bekannte, erfolgreiche Designstrategien zu übertragen
    \end{itemize}
    \item \textbf{Verfication und Validation:} Inspecktion (\textit{code reviews}) und Testen (viele Verschiedene Formen des Testings:
    \begin{itemize}
        \item defect testing
        \item validation testing
        \item unit testing
        \item component testing
        \item system testing
        \item regression testing
        \item \dots
    \end{itemize}
    \item Project Management
    \begin{itemize}
        \item Einhaltung des Zeitplans und des Budgets
        \item \textbf{Risikomanagement} gilt als einer der wichtigsten Punkte, Hauptwerkzeug: Notfallplan
        \item Personalmangement: Bedürfnishierachie, Persönlichkeitstypen
    \end{itemize}
    \item Project planning
    \begin{itemize}
        \item Milestones, deliverables, Gantt (or activity bar) chart
        \item \textbf{Kostenabschätzung COCOMO2}
    \end{itemize}
    \item Quality mangement 
    \begin{itemize}
        \item \textbf{Prüfplan, ISO 9001-Standart, software metrics }
    \end{itemize}
    \item \textbf{CMMI-Modell für Prozessoptimierung}
\end{itemize}
