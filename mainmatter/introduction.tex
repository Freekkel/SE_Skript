\chapter{Introduction to Software Engineering}
\section{Was versteht man unter Software-Engineering?}
Software Engineering ist eine technische Disziplin, die sich mit allen Aspekten der Softwareherstellung beschäftigt.
\subsection{Technische Disziplin:}
\begin{itemize}
    \item Techniker wenden Methoden, Werkzeuge und Theorien an um Dinge zum Laufen zu bringen
    \item Techniker erkennen an, dass sie mit oganisatorischen und finanziellen Beschränkungen arbeiten müssen
\end{itemize}
\subsection{Alle Aspekte der Softwareherstellung:}
\begin{itemize}
    \item SE beschäftigt sich nicht nur mit technischen Prozessen der Softwareentwicklung
    \item auch Projektverwaltung und Entwicklung von Werkzeugen, Methoden und Theorien, welche die Sofwareherstellung unterstützen
\end{itemize}
\subsection{Softwareprozess (grundlegende Aktivitäten von SE):}
\begin{itemize}
    \item Softwarespezifikation
    \begin{itemize}
        \item Kunden und Entwickler definieren die zu produzierende Software und die Rahmenbedingung fr ihren Einsatz
    \end{itemize}
    \item Softwareentwicklung
    \begin{itemize}
        \item Software wird entworfen und programmiert
    \end{itemize}
    \item Softwarevalidierung (Softwarebeurteilung):
    \begin{itemize}
        \item Software wird überprüft um sicherzustellen, dass sie den Kundenanforderungen entspricht
    \end{itemize}
    \item Softwareevolution
    \begin{itemize}
        \item Software wird weiterentwickelt, damit sie die sich ändernden Bedürfnisse der Kunden und des Marktes wiederspiegelt
    \end{itemize}
\end{itemize}

\subsection{Zwei grundlegende Arten von Softwareprodukten:}
\begin{itemize}
    \item Allgemeine Prdoukte:
    \begin{itemize}
        \item eingeständige Systeme die am freien Markt an jeden Kunden verkauft werden kann, der es sich leisten kann
        \item Beispielsweise: Software für den PC, Datenbanken, Textverarbeitungsprogramme
    \end{itemize}
    \item Angepasste Produkte:
    \begin{itemize}
        \item Software die im Auftrag eines bestimmten Kunden hergestellt werden
        \item Biespielsweise: Steuerungssysteme für elektronische Geräte, System zur Unterstützung eines bestimmten Geschäftprozesses
    \end{itemize}
\end{itemize}

\subsection{Unterschiede dieser Softwaretypen:}
\begin{itemize}
    \item Allgemeine Produkte:
    \begin{itemize}
        \item Unternehmen, das Software entwickelt, bestimmt die Spezifikationen der Software
    \end{itemize}
    \item Angepasste Produkte:
    \begin{itemize}
        \item Spezifikation wird von dem Kunden entwickelt und kontrolliert, dass das System kauft
    \end{itemize}
\end{itemize}

\subsection{Viele unterschiedliche Anwendungsarten:}
\begin{itemize}
    \item Eigenständige (stand-alone) Anwendungen:
    \begin{itemize}
        \item Softwware läuft lokal
        \item Beinhaltet alle notwendigen Funktionen um ohne Netzwerkverbindung zu laufen
        \item z.B. Fotoanwendungen ohne Netzwerkzugriff
    \end{itemize}
    \item Interaktive transaktionsbasierte Anwendungen:
    \begin{itemize}
        \item laufen auf externen PC's und werden vom User aufgerufen
        \item Webapplikationen über Cloud-Zugriff
    \end{itemize}
    \item Eingebettete Steuerungssysteme:
    \begin{itemize}
        \item es gibt daon mehr als von allen andern Anwendungsarten
        \item Software für ABS im Auto
        \item Kontrollieren die Hardware
    \end{itemize}
    \item Stapelverarbeitende Systeme (Batch-Verarbeitungssysteme)
    \begin{itemize}
        \item Systeme die große Datenmengen verarbeiten (z.B. Business-Systeme)
        \item Abrechnung von regelmäßigen Zahlungen (Telefonrechnung)
    \end{itemize}
    \item Unterhaltungssysteme
    \begin{itemize}
        \item zum persönlichen Gebrauch
        \item z.B. Spiele
    \end{itemize}
    \item Systeme für die Modellierung und Simulation
    \begin{itemize}
        \item Systeme, die von Wissenschaftlern und Ingenieuren entwickelt wurden
        \item z.B. physikalische Prozesse simulieren
    \end{itemize}
    \item Datenerfassungssysteme
    \begin{itemize}
        \item System, welches Daten unter extremen Bedingungen mit Hilfe von Sensoren erfassen und weiterleiten
    \end{itemize}
    \item Systeme von Systemen
    \begin{itemize}
        \item Zusammensetzung von vielen Softwaresystemen
    \end{itemize}
\end{itemize}

\subsection{Wesentliche Merkmale guter Software:}
\begin{itemize}
    \item Wartbarkeit
    \begin{itemize}
        \item kritisches Merkmal 
        \item Software sollte so geschrieben werden, dass sie \textbf{weiterentwickelt werden kann}, um sich verändernden Kundenbedüfnissen Rechnung zu tragen
        \item Softwareanpassungen sind eine unvermeidliche Anforderung einer sich verändernden Geschäftsumgebung 
    \end{itemize}
    \item Verlässlichkeit (dependability) und Informationssicherheit (security):
    \begin{itemize}
        \item Verlässliche Software sollte keinen körperlichen oder wirtschaftlichen Schaden verursachen, falls das System ausfällt
        \item Böswillige Benutzer sollten nicht in der Lage sein, auf das System zuzugreifen oder es zu beschädigen
    \end{itemize}
    \item Effizienz (efficiency):
    \begin{itemize}
        \item Software sollte nicht verschwenderisch mit Systemressourcen wie Speicher und Prozessortkapazität umgehen
        \item Effizien umfasst somit Reaktionszeit, Verarbeitungszeit, Speichernutzung usw. 
    \end{itemize}
    \item Akzeptanz (acceptability)
    \begin{itemize}
        \item Software muss von den Benutzern akzeptiert werden, für die sie entwickelt wurde. Das bedeutet, dass sie verständlich, nützlich und kompatibel mit anderen Systemen sein müssen, die diese Benutzer verwenden.
    \end{itemize}
\end{itemize}
